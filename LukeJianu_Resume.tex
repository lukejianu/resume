\documentclass{resume}

\usepackage[left=0.75in,top=0.51in,right=0.75in,bottom=0.51in]{geometry}
\usepackage{hyperref}

\name{Luke Jianu}
\address{{+1 (425) 229-1106} \\ 
        {jianuluke@gmail.com} \\
        \href{https://github.com/lukejianu}{github/lukejianu} \\
        \href{https://www.linkedin.com/in/lukejianu/}{linkedin/lukejianu}}

\def\nameskip{\bigskip}
\def\sectionskip{\medskip}

\begin{document}
  \begin{rSection}{Education}
    \begin{rSubsection}{Northeastern University}{Sept. 2021 - May 2025}
      {\normalfont B.S. in Computer Science, 4.00/4.00 GPA}{Boston, MA}
      \begin{tabular}{ @{} >{\bfseries}l @{\hspace{6ex}} l }
          \emph{Relevant} & Object-Oriented Design, Algorithms \& Data, Logic \& Computation \\
          \emph{Coursework} & Database Design, Foundations of Data Science, Foundations of Cybersecurity \\
      \end{tabular} 
    \end{rSubsection}
  \end{rSection}
  \begin{rSection}{Experience}

    \begin{rSubsection}{Belvedere Trading}{Jun. 2023 - Aug. 2023}{Software Engineer Intern}{Chicago, IL}
      \item Built a low-latency service-agnostic proxy in C++ to aggregate redundant TCP
        connections between datacenters, resulting in a \textbf{70\% reduction} in bandwidth usage for proxied services.
      \item Optimized performance through the use of asynchronous message passing, implemented with
        the visitor design pattern (std::variant), enabling the processing of \textbf{5.4Tb of data} daily.
      \item Modified the service discovery algorithm in C\# to match clients with services in the same datacenter.
    \end{rSubsection}

    \begin{rSubsection}{Amazon Robotics}{Jan. 2023 - Jun. 2023}{Software Development Engineer Co-op}{North Reading, MA}
      \item Empowered AR teams to rapidly grow, monitor and manage their device fleets at scale by inventing and simplifying
        features in my team's 
        \textbf{\href{https://www.allthingsdistributed.com/2021/07/amazon-robotics-on-aws.html}{Comprehensive Device Management}} solution. 
      \item Architected a move device system in AWS Lambda with Kotlin, enabling teams to provision robotic workcells in test 
        sites, saving \textbf{1000+ hours} of software installs yearly for IT at production sites.
      \item Designed a device timeline feature in AWS CDK with TypeScript, providing insights for \textbf{30k+ devices}. 
      \item Refactored a large, imperative-style vanilla React codebase with functional-style 
        TypeScript \& React Query, resulting in a \textbf{75\% reduction} in API calls and \textbf{50\% faster} loading times.
    \end{rSubsection}

    \begin{rSubsection}{S3Global}{May 2022 - Aug. 2022}{Software Development Intern}{Redmond, WA}
      \item Created a .NET WPF application to interface with \textbf{12 \href{https://emergentvisiontec.com/}{high-speed cameras}}, 
        including functionality to toggle recording, change scale and video RGB, and reboot bugged cameras.
      \item Implemented streaming using C++ interop with the camera's SDK to dump frames into shared buffers.
    \end{rSubsection}

  \end{rSection}
  
  \begin{rSection}{Projects}

    \begin{rSubsection}{NUCarpool}{Sept. 2022 - Present}{Software Developer at Sandbox}{Boston, MA}
      \item Developing a web app using the T3 stack for Northeastern students to find co-op carpool groups.
    \end{rSubsection}

    \begin{rSubsection}{Trading Bot}{Jul. 2022}{Participant at Jane Street's Electronic Trading Challenge}{Seattle, WA}
      \item Placed \textbf{10\textsuperscript{th}} by quickly programming penny-pinching and ADR/ETF arbitrage strategies in Python.
    \end{rSubsection}

   \end{rSection}
  \begin{rSection}{Technical Skills}
    \begin{tabular}{ @{} >{\bfseries}l @{\hspace{6ex}} l }
      Computer Languages & C++, Java, Python, JavaScript, TypeScript, SQL \\
      Tools \& Technologies & Vim, Git, AWS, Docker, React, Prisma
    \end{tabular}
  \end{rSection}
\end{document}

